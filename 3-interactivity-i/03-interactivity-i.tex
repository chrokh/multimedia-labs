\documentclass[12pt]{article}
\usepackage{fullpage}
\usepackage[swedish]{babel}
\usepackage[utf8]{inputenc} % åäö
%\usepackage[T1]{fontenc}
\usepackage{graphicx}
\usepackage{hyperref}
\usepackage{xcolor}
\usepackage{listings}
\usepackage{enumitem}



% No numbering
\setcounter{secnumdepth}{0}

% Counters for tasks & questions
\newcounter{taskcounter}
\setcounter{taskcounter}{0}
\newcounter{stepcounter}
\setcounter{stepcounter}{0}
\newcounter{questioncounter}
\setcounter{questioncounter}{0}

% Exercises
\newcommand{\exercise}[1]{
  \refstepcounter{taskcounter}
  \addcontentsline{toc}{subsection}{Uppgift \thetaskcounter{} #1}
  \vspace{1em}~
  \\\normalfont{\large{\bfseries{\hspace{0.5em}Uppgift \thetaskcounter \hspace{1em}#1}}}\\\\
}

% Remove date
\date{}

\hypersetup{
  colorlinks = true,
  linkcolor = blue,
  citecolor = red
}

\lstset{
  language=[Sharp]C,
  basicstyle=\color[rgb]{0.3,0.3,0.3}\ttfamily,
  keywordstyle=\color[rgb]{0,0.5,0.5},
  numberstyle=\color[rgb]{0.7,0.7,0.7},
  commentstyle=\color[rgb]{0.1,0.5,0.1},
  stringstyle=\color[rgb]{0.6,0.1,0.5},
  backgroundcolor=\color[rgb]{0.95,0.95,0.95},
  showstringspaces=false,
  numbers=left,
  breaklines,
  breakatwhitespace,
}


\title{ Labb 2 -- Pixelgrafik }

\author{ Multimedia 7.5 hp VT-14 }
\begin{document}
\maketitle
\vspace{-2em}
%\tableofcontents



\section{Introduktion}
I denna laboration kommer vi att arbeta med javascript för att skapa interaktivitet. Tanken med laborationen är att vi kommer att befästa och lära oss javascript, html och css parallelt. Vi förväntar oss att ni har goda kunskaper i programmering och att inläsning på iteration, selektion, variabler och metoder görs innan. 

\section{Inlämning}
Denna laboration består av tre faser där du ska lämna in varje fas i en egen mapp. Din inlämning ska alltså bestå av en .zip-fil eller .rar-fil (inga andra komprimeringsformat är tillåtna!) innehållandes följande (med följande struktur):
  \begin{itemize}
    \item labb2\_fornamn\_efternamn.zip

      \begin{itemize}
        \item uppgift1 (mapp)
          \begin{itemize}
            \item Hello.js 
            \item index.html
      
          \end{itemize}

        \item uppgift2 (mapp)
          \begin{itemize}
	    \item compositeImage.psd (Ditt kollage)
            \item compositeImage.jpg (Ovan exporterad till .jpg)
            \item originalPicture1.jpg (Första orginalbilden)
            \item originalPicture2.jpg (Andra orginalbilden)
          \end{itemize}
	\item uppgift3 (mapp)
          \begin{itemize}
	    \item compositeImage.psd (Ditt kollage)
            \item compositeImage.jpg (Ovan exporterad till .jpg)
            \item originalPicture1.jpg (Första orginalbilden)
            \item originalPicture2.jpg (Andra orginalbilden)
          \end{itemize}
    \end{itemize}
  \end{itemize}


\pagebreak
\section{Uppgifter}
Nedan följer uppgifterna som resulterar i inlämningarna ovan.



  \exercise{Hej! \emph{Namn}}
 I denna uppgift skall ni skapa ett formulär i HTML som kan med hjälp av javascript läsa in från textboxen och sedan säga Hej + \emph {textboxens värde} 
Alltså kommer ert formulär se ut som nedanstående: 
\\ \\
Skriv in ditt namn: \fcolorbox{black}{white}{ \emph Namn...} \\ 
\colorbox{gray}{ \large\bf OK }
\\

  \begin{enumerate}
    \item Börja med att skapa en .html-fil där du skapar upp ett skal för din html. 
    \item Skapa sedan en .js-fil där ni kommer att skapa metoder som ni behöver. 
    \item Skapa sedan en .cc-fil där all formatering skall finnas! 
    \item Försök nu skapa de element ni behöver, alltså ett inputfält och en knapp.  
    \item Försök nu att endast få er knapp att skriva ut något på skärmen, förslagsvis med en alert()
   \item nästa steg är att försöka läsa in från textboxen och få ut detta i en alertbox. Klar! 
 \end{enumerate}


  \paragraph{Tips och tricks}
    \begin{itemize}
      \item För att skapa din kod direkt i webbläsaren istället för filerna använd: \href{http://jsfiddle.net/}{jsfiddle}. Observera dock att ni sedan måste länka ihop era filer senare.
      \item 
    \end{itemize}






  \exercise{Bakgrundsrandomiserare}
  Denna övning går ut på att göra (nästan) exakt samma sak som i förra övningen, fast nu genom att använda programmet \emph{Adobe Photoshop}.


\exercise{Interaktiv meny}
  Du bör nu kunna frilägga ett specifikt objekt från en bild och lägga in den i en
  ny bild dvs göra ett montage. 






\end{document}