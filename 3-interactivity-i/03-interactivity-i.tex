\documentclass[12pt]{article}
\usepackage{fullpage}
\usepackage[swedish]{babel}
\usepackage[utf8]{inputenc} % åäö
%\usepackage[T1]{fontenc}
\usepackage{graphicx}
\usepackage{hyperref}
\usepackage{xcolor}
\usepackage{listings}
\usepackage{enumitem}

% No numbering
\setcounter{secnumdepth}{0}

% Counters for tasks & questions
\newcounter{taskcounter}
\setcounter{taskcounter}{0}
\newcounter{stepcounter}
\setcounter{stepcounter}{0}
\newcounter{questioncounter}
\setcounter{questioncounter}{0}

% Exercises
\newcommand{\exercise}[1]{
  \refstepcounter{taskcounter}
  \addcontentsline{toc}{subsection}{Uppgift \thetaskcounter{} #1}
  \vspace{1em}~
  \\\normalfont{\large{\bfseries{\hspace{0.5em}Uppgift \thetaskcounter \hspace{1em}#1}}}\\\\
}

% Remove date
\date{}

\hypersetup{
  colorlinks = true,
  linkcolor = blue,
  citecolor = red
}

\lstset{
  language=[Sharp]C,
  basicstyle=\color[rgb]{0.3,0.3,0.3}\ttfamily,
  keywordstyle=\color[rgb]{0,0.5,0.5},
  numberstyle=\color[rgb]{0.7,0.7,0.7},
  commentstyle=\color[rgb]{0.1,0.5,0.1},
  stringstyle=\color[rgb]{0.6,0.1,0.5},
  backgroundcolor=\color[rgb]{0.95,0.95,0.95},
  showstringspaces=false,
  numbers=left,
  breaklines,
  breakatwhitespace,
}

\title{ Labb 2 -- Pixelgrafik }

\author{ Multimedia 7.5 hp VT-14 }
\begin{document}
\maketitle
\vspace{-2em}
%\tableofcontents



\section{Introduktion}
I denna laboration kommer vi att arbeta med pixelgrafik ur två aspekter. Först genom att manuellt skapa pixelgrafik med hjälp av HTML-elementet canvas och sedan genom att arbeta med programmet \emph{Adobe Photoshop}.

\section{Inlämning}
Denna laboration består av tre faser där du ska lämna in varje fas i en egen mapp. Din inlämning ska alltså bestå av en .zip-fil eller .rar-fil (inga andra komprimeringsformat är tillåtna!) innehållandes följande (med följande struktur):
  \begin{itemize}
    \item labb2\_fornamn\_efternamn.zip

      \begin{itemize}
        \item uppgift1 (mapp)
          \begin{itemize}
            \item uppgift.js (din)
            \item pattern1.png (som du utgick ifrån)
            \item pattern2.png (som du utgick ifrån)
            \item pattern3.png (som du utgick ifrån)
          \end{itemize}

        \item uppgift2 (mapp)
          \begin{itemize}
	    \item compositeImage.psd (Ditt kollage)
            \item compositeImage.jpg (Ovan exporterad till .jpg)
            \item originalPicture1.jpg (Första orginalbilden)
            \item originalPicture2.jpg (Andra orginalbilden)
          \end{itemize}
	\item uppgift3 (mapp)
          \begin{itemize}
	    \item compositeImage.psd (Ditt kollage)
            \item compositeImage.jpg (Ovan exporterad till .jpg)
            \item originalPicture1.jpg (Första orginalbilden)
            \item originalPicture2.jpg (Andra orginalbilden)
          \end{itemize}
    \end{itemize}
  \end{itemize}


\pagebreak
\section{Uppgifter}
Nedan följer uppgifterna som resulterar i inlämningarna ovan.



  \exercise{Hello <name>}
  Denna övning går ut på att göra (nästan) exakt samma sak som i förra övningen, fast nu genom att använda programmet \emph{Adobe Photoshop}.


  \exercise{Bakgrundsrandomiserare}
  Denna övning går ut på att göra (nästan) exakt samma sak som i förra övningen, fast nu genom att använda programmet \emph{Adobe Photoshop}.


\exercise{Interaktiv meny}
  Du bör nu kunna frilägga ett specifikt objekt från en bild och lägga in den i en
  ny bild dvs göra ett montage. 






\end{document}