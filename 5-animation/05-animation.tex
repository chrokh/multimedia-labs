\documentclass[12pt]{article}
\usepackage{fullpage}
\usepackage[swedish]{babel}
\usepackage[utf8]{inputenc} % åäö
%\usepackage[T1]{fontenc}
\usepackage{graphicx}
\usepackage{hyperref}
\usepackage{xcolor}
\usepackage{listings}
\usepackage{enumitem}



% No numbering
\setcounter{secnumdepth}{0}

% Counters for tasks & questions
\newcounter{taskcounter}
\setcounter{taskcounter}{0}
\newcounter{stepcounter}
\setcounter{stepcounter}{0}
\newcounter{questioncounter}
\setcounter{questioncounter}{0}

% Exercises
\newcommand{\exercise}[1]{
  \refstepcounter{taskcounter}
  \addcontentsline{toc}{subsection}{Uppgift \thetaskcounter{} #1}
  \vspace{1em}~
  \\\normalfont{\large{\bfseries{\hspace{0.5em}Uppgift \thetaskcounter \hspace{1em}#1}}}\\\\
}

% Remove date
\date{}

\hypersetup{
  colorlinks = true,
  linkcolor = blue,
  citecolor = red
}

\lstset{
  language=[Sharp]C,
  basicstyle=\color[rgb]{0.3,0.3,0.3}\ttfamily,
  keywordstyle=\color[rgb]{0,0.5,0.5},
  numberstyle=\color[rgb]{0.7,0.7,0.7},
  commentstyle=\color[rgb]{0.1,0.5,0.1},
  stringstyle=\color[rgb]{0.6,0.1,0.5},
  backgroundcolor=\color[rgb]{0.95,0.95,0.95},
  showstringspaces=false,
  numbers=left,
  breaklines,
  breakatwhitespace,
}


\title{ Labb 5 -- Animation }

\author{ Multimedia 7.5 hp VT-14 }
\begin{document}
\maketitle
\vspace{-3.5em}
%\tableofcontents



\section{Introduktion}
I denna laboration kommer vi att arbeta med animation.

\section{Inlämning}
Din inlämning ska bestå av en .zip-fil eller .rar-fil (inga andra komprimeringsformat är tillåtna!) innehållandes följande (med följande struktur):
  \begin{itemize}
    \item labb5\_fornamn\_efternamn.zip
      \vspace{-0.5em}
      \begin{itemize}
        \item uppgift1 (mapp)
          \begin{itemize}
            \item index.html
            \item transformations.css
          \end{itemize}
        \item uppgift2 (mapp)
          \begin{itemize}
            \item index.html
            \item transitions.css
          \end{itemize}
        \item uppgift3 (mapp)
          \begin{itemize}
            \item index.html
            \item animations.css
          \end{itemize}
    \end{itemize}
  \end{itemize}



\section{Uppgifter}
Nedan följer uppgifterna som resulterar i inlämningarna ovan.




  \exercise{CSS3 Transformations}
  TODO

  \exercise{CSS3 Transitions}
  TODO

  \exercise{CSS3 Animations}
  TODO


\end{document}