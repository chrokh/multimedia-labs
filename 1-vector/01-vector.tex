\documentclass[12pt]{article}
\usepackage{fullpage}
\usepackage[swedish]{babel}
\usepackage[utf8]{inputenc} % åäö
%\usepackage[T1]{fontenc}
\usepackage{graphicx}
\usepackage{hyperref}
\usepackage{xcolor}
\usepackage{listings}
\usepackage{enumitem}

% No numbering
\setcounter{secnumdepth}{0}

% Counters for tasks & questions
\newcounter{taskcounter}
\setcounter{taskcounter}{0}
\newcounter{stepcounter}
\setcounter{stepcounter}{0}
\newcounter{questioncounter}
\setcounter{questioncounter}{0}

% Exercises
\newcommand{\exercise}[1]{
  \refstepcounter{taskcounter}
  \addcontentsline{toc}{subsection}{Uppgift \thetaskcounter{} #1}
  \vspace{1em}~
  \\\normalfont{\large{\bfseries{\hspace{0.5em}Uppgift \thetaskcounter \hspace{1em}#1}}}\\\\
}

% Remove date
\date{}

\hypersetup{
  colorlinks = true,
  linkcolor = blue,
  citecolor = red
}

\lstset{
  language=[Sharp]C,
  basicstyle=\color[rgb]{0.3,0.3,0.3}\ttfamily,
  keywordstyle=\color[rgb]{0,0.5,0.5},
  numberstyle=\color[rgb]{0.7,0.7,0.7},
  commentstyle=\color[rgb]{0.1,0.5,0.1},
  stringstyle=\color[rgb]{0.6,0.1,0.5},
  backgroundcolor=\color[rgb]{0.95,0.95,0.95},
  showstringspaces=false,
  numbers=left,
  breaklines,
  breakatwhitespace,
}

\title{ Labb 1 -- Vektorgrafik }

\author{ Multimedia 7.5 hp VT-14 }
\begin{document}
\maketitle
\vspace{-2em}
%\tableofcontents



\section{Introduktion}
I denna laboration kommer vi att arbeta med vektorgrafik ur två aspekter. Först genom att manuellt skapa .svg-grafik och sedan genom att arbeta med programmet \emph{Adobe Illustrator}.

\section{Inlämning}
Denna laboration består av tre faser där du ska lämna in varje fas i en egen mapp. Din inlämning ska alltså bestå av en .zip-fil eller .rar-fil (inga andra komprimeringsformat är tillåtna!) innehållandes följande (med följande struktur):
  \begin{itemize}
    \item labb1\_fornamn\_efternamn.zip

      \begin{itemize}
        \item uppgift1 (mapp)
          \begin{itemize}
            \item pattern.svg (din svg-fil)
            \item goal.png (den bild du utgick ifrån)
          \end{itemize}

        \item uppgift2 (mapp)
          \begin{itemize}
            \item pattern.ai (Ditt mönster skapat i illustrator)
            \item pattern.png (Ovan dokument exporterat till .png)
            \item goal.png
          \end{itemize}

        \item uppgift3 (mapp)
          \begin{itemize}
            \item tools.ai (Ditt dokument som använder sig av alla tekinker i checklistan)
            \item tools.png (Ovan dokument exporterat till .png)
          \end{itemize}

    \end{itemize}

  \end{itemize}


\pagebreak
\section{Uppgifter}
Nedan följer uppgifterna som resulterar i inlämningarna ovan.

  \exercise{Enkla mönster i SVG}
  Denna övning går ut på att försöka återskapa ett befintligt grafiskt mönster genom att använda sig av tekniken SVG.

  \begin{enumerate}
    \item Börja med att välja ett mönster på sidan \href{http://chrokh.github.io/svg-and-canvas-exercises}{http://chrokh.github.io/svg-and-canvas-exercises}. \textbf{Välj inte samma mönster som någon bredvid dig.} Spara ned en kopia av mönstret så att du inte tappar bort den och kan ha den som referens.
    \item Alla mönster är 500x500 pixlar, så tänk på det när du börjar med nästa steg.
    \item Försök nu återskapa mönstret genom att använda SVG. Du kan arbeta på det sätt som du finner enklast. Men vår rekommendation är att du använder ett online-verktyg såsom t.ex. \href{http://www.w3schools.com/html/tryit.asp?filename=tryhtml5_svg_ex}{tryit ifrån W3Scools} eller \href{http://jsfiddle.net/}{jsfiddle}.
    \item När du lyckats återskapa mönstret så behöver du spara ditt svg-element i ett separat dokument (som du namnger enligt inlämningsinstruktionerna). Du ska alltså \textbf{endast spara och lämna in ditt svg-element} och inte hela html-dokumentet.

    Exempel på korrekt inlämning (\texttt{filnamn.svg}):
    \begin{lstlisting}
<svg width="500" height="500">
  <rect x="100" y="100" width="300" height="300" fill="rgb(255,0,0)"/>
</svg>
    \end{lstlisting}
  \end{enumerate}






  \exercise{Enkla mönster i Illustrator}
  Denna övning går ut på att göra (nästan) exakt samma sak som i förra övningen, fast nu genom att använda programmet \emph{Adobe Illustrator}.

    \begin{enumerate}
      \item Välj ett nytt mönster.
      \item Skapa ett nytt Illustrator-dokument som är 500x500 pixlar stort.
      \item Återskapa det andra mönstret genom att använda verktygen i Illustrator.
      \item Ge varje ``färgfält'' i din fil en toning. Med andra ord: Om du har ett rött färgfält, gör istället så att det går ifrån mörkrött till ljusrött.
    \end{enumerate}






  \exercise{Övningar i Illustrator}
  Denna sista övning går ut på att träna på olika tekniker i \emph{Adobe Illustrator}. Vilken bild du väljer att skapa är helt upp till dig, men uppgiften kräver att alla tekniker/moment som nämns i nedan checklista måste ingå. Utöver att du behöver använda alla teknikerna så behöver du lägga in textfält som uppenbart visar vilken teknik du använt vart.

  \paragraph{Checklistan}
    \begin{itemize}
      \item Minst 4 olika geometriska figurer (ex. cirkel, rektangel, stjärna etc.)
      \item Minst 1 fri Beizér-kurva (genom ex. ritstiftet/pen tool) utan fyllning.
      \item Minst 1 fri Beizér-kurva med fyllning.
      \item Minst en gradient
      \item Minst 1 text som följer en bana/kurva.
    \end{itemize}



\end{document}