\documentclass[12pt]{article}
\usepackage{fullpage}
\usepackage[swedish]{babel}
\usepackage[utf8]{inputenc} % åäö
%\usepackage[T1]{fontenc}
\usepackage{graphicx}
\usepackage{hyperref}
\usepackage{xcolor}
\usepackage{listings}
\usepackage{enumitem}



% No numbering
\setcounter{secnumdepth}{0}

% Counters for tasks & questions
\newcounter{taskcounter}
\setcounter{taskcounter}{0}
\newcounter{stepcounter}
\setcounter{stepcounter}{0}
\newcounter{questioncounter}
\setcounter{questioncounter}{0}

% Exercises
\newcommand{\exercise}[1]{
  \refstepcounter{taskcounter}
  \addcontentsline{toc}{subsection}{Uppgift \thetaskcounter{} #1}
  \vspace{1em}~
  \\\normalfont{\large{\bfseries{\hspace{0.5em}Uppgift \thetaskcounter \hspace{1em}#1}}}\\\\
}

% Remove date
\date{}

\hypersetup{
  colorlinks = true,
  linkcolor = blue,
  citecolor = red
}

\lstset{
  language=[Sharp]C,
  basicstyle=\color[rgb]{0.3,0.3,0.3}\ttfamily,
  keywordstyle=\color[rgb]{0,0.5,0.5},
  numberstyle=\color[rgb]{0.7,0.7,0.7},
  commentstyle=\color[rgb]{0.1,0.5,0.1},
  stringstyle=\color[rgb]{0.6,0.1,0.5},
  backgroundcolor=\color[rgb]{0.95,0.95,0.95},
  showstringspaces=false,
  numbers=left,
  breaklines,
  breakatwhitespace,
}


\title{ Labb 7 -- Video }

\author{ Multimedia 7.5 hp VT-14 }
\begin{document}
\maketitle
\vspace{-3.5em}
%\tableofcontents


% FÖRBEREDELSE-MATERIAL
% http://www.youtube.com/watch?v=VJufqV6P-xY (Scrolling background header)


\section{Introduktion}
I denna laboration kommer vi att arbeta med författande och redigering av video, såväl som inbäddning av video i webbsidor.

\section{Inlämning}
Din inlämning ska bestå av en .zip-fil eller .rar-fil (inga andra komprimeringsformat är tillåtna!) innehållandes följande (med följande struktur):
  \begin{itemize}
    \item \texttt{labb7\_fornamn\_efternamn.zip}
      \vspace{-0.5em}
      \begin{itemize}
        \item \texttt{uppgift1} (mapp)
          \begin{itemize}
            \item \texttt{movie.mp4}
            \item \texttt{movie.ogg}
          \end{itemize}
        \item \texttt{uppgift2} (mapp)
          \begin{itemize}
            \item \texttt{index.html}
            \item \texttt{video.css}
          \end{itemize}
    \end{itemize}
  \end{itemize}
  \paragraph{}
  Under denna labb är det ok att arbeta i grupp (om \textbf{max 2 personer}). Om ni har arbetat i grupp ska den ena ladda upp inlämningen och den andra en fil som innehåller namnet på den ni har arbetat med. Filen ska heta:
  \begin{itemize}
    \item \texttt{samarbetspartner.txt}
  \end{itemize}

  \subsection{Krav}
  \begin{itemize}
    \item Videofilerna \texttt{movie.[mp4/ogg]} skall alltså \textbf{endast} lämnas in i \texttt{uppgift 1}. I \texttt{uppgift 2} ska du referera till dina videofiler som ligger i mappen \texttt{uppgift1} genom \texttt{relativ länkning}.
    \item All kod ska vara korrekt indenterad. Läs mer på \href{http://htmlhunden.se}{HTMLHunden} om du är osäker på hur man indenterar.
    \item Dina exporterade videofiler måste vara anpassade för webben, i termer av filstorlek, kvalitet och dimensioner.
  \end{itemize}



\pagebreak
\section{Uppgifter}
Nedan följer uppgifterna som resulterar i inlämningarna ovan.




  \exercise{Videoförfatttning -- iMovie}
  Denna uppgift går ut på att skapa en egen filmtrailer. Detta genom att arbeta med antingen i egeninspelat eller redan existerande material.

  \begin{enumerate}
    \item Börja med att skaffa dig ``råmaterial'' att arbeta med. Antingen genom att:
      \begin{itemize}
        \item Spela in klipp m.h.a. \texttt{Photo Booth}, eller
        \item Ladda hem klipp ifrån internet (t.ex. genom att använda dig av en tjänst som möjliggör nedladdning av \texttt{YouTube}-klipp)
      \end{itemize}
    \item Använd sedan iMovie för att ``klippa och klistra'' i materialet tills du uppfyllt alla nedan krav och känner dig nöjd med trailern.
  \end{enumerate}

  \paragraph{Krav}
  \begin{itemize}
    \item Det ska vara tydligt att din film är en trailer/teaser för någonting.
    \item Trailern får vara \textbf{max 2 min} lång.
    \item Trailern måste innehålla...
      \begin{itemize}
        \item minst en vinjett med era namn,
        \item minst en stillbild som zoomas eller förflyttas på något sätt,
        \item övergångseffekter (\texttt{transitions}), och
        \item text.
      \end{itemize}
  \end{itemize}

  \paragraph{Frivilligt}
  Om du vill får du även arbeta med ljud i din trailer. Se dock då till att du även komprimerar ljudet när du exporterar filmen.


  \pagebreak
  \exercise{Videodistribution -- HTML5}
  Vi ska nu prova på ett sätt att bädda in video i webbsidor. Detta genom att använda oss av \texttt{HTML5}-elementet \texttt{<video>}.

  \begin{enumerate}
    \item Skapa en \texttt{.html}-, en \texttt{.css}-fil och ``koppla ihop dem''.
    \item Använd \texttt{<video>}-elementet för att bädda in dina videofiler.
  \end{enumerate}

  \paragraph{Krav}
  \begin{itemize}
    \item Kom ihåg att ge båda versionerna (formaten) av din videofil som alternativ till \texttt{<video>}-elementet. Alltså både \texttt{.mp4} och \texttt{.ogg}.
    \item Kom ihåg att du inte ska kopiera dina videofiler ifrån \texttt{uppgift 1} till denna uppgift. Utan istället ska du använda \texttt{relativa url}:er för att länka till filerna i mappen för \texttt{uppgift 1}.
  \end{itemize}


\end{document}