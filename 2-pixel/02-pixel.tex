\documentclass[12pt]{article}
\usepackage{fullpage}
\usepackage[swedish]{babel}
\usepackage[utf8]{inputenc} % åäö
%\usepackage[T1]{fontenc}
\usepackage{graphicx}
\usepackage{hyperref}
\usepackage{xcolor}
\usepackage{listings}
\usepackage{enumitem}

% No numbering
\setcounter{secnumdepth}{0}

% Counters for tasks & questions
\newcounter{taskcounter}
\setcounter{taskcounter}{0}
\newcounter{stepcounter}
\setcounter{stepcounter}{0}
\newcounter{questioncounter}
\setcounter{questioncounter}{0}

% Exercises
\newcommand{\exercise}[1]{
  \refstepcounter{taskcounter}
  \addcontentsline{toc}{subsection}{Uppgift \thetaskcounter{} #1}
  \vspace{1em}~
  \\\normalfont{\large{\bfseries{\hspace{0.5em}Uppgift \thetaskcounter \hspace{1em}#1}}}\\\\
}

% Remove date
\date{}

\hypersetup{
  colorlinks = true,
  linkcolor = blue,
  citecolor = red
}

\lstset{
  language=[Sharp]C,
  basicstyle=\color[rgb]{0.3,0.3,0.3}\ttfamily,
  keywordstyle=\color[rgb]{0,0.5,0.5},
  numberstyle=\color[rgb]{0.7,0.7,0.7},
  commentstyle=\color[rgb]{0.1,0.5,0.1},
  stringstyle=\color[rgb]{0.6,0.1,0.5},
  backgroundcolor=\color[rgb]{0.95,0.95,0.95},
  showstringspaces=false,
  numbers=left,
  breaklines,
  breakatwhitespace,
}

\title{ Labb 2 -- Pixelgrafik }

\author{ Multimedia 7.5 hp VT-14 }
\begin{document}
\maketitle
\vspace{-2em}
%\tableofcontents



\section{Introduktion}
I denna laboration kommer vi att arbeta med pixelgrafik ur två aspekter. Först genom att manuellt skapa pixelgrafik med hjälp av HTML-elementet \texttt{<canvas>} och sedan genom att arbeta med programmet \emph{Adobe Photoshop}.

\section{Inlämning}
Denna laboration består av tre faser där du ska lämna in varje fas i en egen mapp. Din inlämning ska alltså bestå av en .zip-fil eller .rar-fil (inga andra komprimeringsformat är tillåtna!) innehållandes följande (med följande struktur):
  \begin{itemize}
    \item labb2\_fornamn\_efternamn.zip

      \begin{itemize}
        \item uppgift1 (mapp)
          \begin{itemize}
            \item patterns.js (dina färdigskrivna funktioner)
            \item original1.png (som du utgick ifrån)
            \item original2.png (som du utgick ifrån)
            \item original3.png (som du utgick ifrån)
          \end{itemize}

        \item uppgift2 (mapp)
          \begin{itemize}
            \item pattern.psd  (ditt mönster återskapat i \emph{Photoshop})
            \item pattern.png  (ovan exporterad till .png)
            \item original.png (som du utgick ifrån)
          \end{itemize}

        \item uppgift3 (mapp)
          \begin{itemize}
	    \item compositeImage.psd (ditt montage)
            \item compositeImage.jpg (ovan exporterad till .jpg)
            \item originalPicture1.jpg (första orginalbilden)
            \item originalPicture2.jpg (andra orginalbilden)
          \end{itemize}
    \end{itemize}
  \end{itemize}


\pagebreak
\section{Uppgifter}
Nedan följer uppgifterna som resulterar i inlämningarna ovan.




\exercise{Rita med hjälp av javascript och canvas}
  För att underlätta arbetet med denna labb, samt för att skapa förståelse för i vilken kontext vi använder oss av canvas, så får ni ett skalprojekt att utgå ifrån.

  \begin{enumerate}
      \item \href{https://github.com/BinaryHeart/canvas-viewer/archive/master.zip}{Ladda ned skalprojektet} och undersök fil-/mappstrukturen.
      \item Öppna nu \texttt{index.html} i en webbläsare och begrunda att det finns 3 rutor längst ned. När du klickar på någon av rutorna, visas det som ritats i den lilla rutan i den stora rutan ovan.
      \item Öppna nu istället \texttt{js/patterns.js} i en texteditor (såsom Notepad++, Sublime Text 2, Text Wrangler eller dyl.). Denna fil innehåller tre funktioner. Det är dessa tre funktioner som ritar ut de svarta trianglarna i de tre rutorna du tidigare såg i webbläsaren.
      \item Välj (precis som i labb 1) tre mönster (och spara ned dem på datorn) ifrån \\ \href{http://chrokh.github.io/svg-and-canvas-exercises}{http://chrokh.github.io/svg-and-canvas-exercises}.
      \item Din uppgift är nu att försöka återskapa dessa tre mönster genom att ersätta koden i de tre funktionerna i filen \texttt{js/patterns.js}.
    \end{enumerate}

  \exercise{Enkla mönster i Photoshop}
  Denna övning går ut på att göra (nästan) exakt samma sak som i förra övningen, fast nu genom att använda programmet \emph{Adobe Photoshop}.

    \begin{enumerate}
      \item Välj ett (det räcker med ett) nytt mönster ifrån \\ \href{http://chrokh.github.io/svg-and-canvas-exercises}{http://chrokh.github.io/svg-and-canvas-exercises}.
      \item Skapa ett nytt \emph{Photoshop}-dokument som är 400x400 pixlar stort.
      \item Återskapa det andra mönstret genom att använda verktygen i \emph{Photoshop}.
    \end{enumerate}

\pagebreak
\exercise{Ett montage i Photoshop!}
  Vi ska nu skapa ett montage i \emph{Photoshop}. Du bör (genom att tillgodosett dig förberedelsematerialet) vara bekant med hur man frilägger ett objekt ifrån en bild och läger in den i en annan m.h.a. \emph{Photoshop} -- alltså göra ett montage.
  \begin{enumerate}
      \item Ladda hem en bild från internet, det finns arkiv med fria bilder. Alternatitv ta egna foton och använd dem.
      \item Klipp ut minst ett objekt ur ena bilden och lägg in den på den andra bilden. Resultatet skall bli sådant att en ovan användare inte skall kunna se att du manipulerat bilden.
    \end{enumerate}





\end{document}