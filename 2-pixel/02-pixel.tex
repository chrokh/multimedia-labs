\documentclass[12pt]{article}
\usepackage{fullpage}
\usepackage[swedish]{babel}
\usepackage[utf8]{inputenc} % åäö
%\usepackage[T1]{fontenc}
\usepackage{graphicx}
\usepackage{hyperref}
\usepackage{xcolor}
\usepackage{listings}
\usepackage{enumitem}

% No numbering
\setcounter{secnumdepth}{0}

% Counters for tasks & questions
\newcounter{taskcounter}
\setcounter{taskcounter}{0}
\newcounter{stepcounter}
\setcounter{stepcounter}{0}
\newcounter{questioncounter}
\setcounter{questioncounter}{0}

% Exercises
\newcommand{\exercise}[1]{
  \refstepcounter{taskcounter}
  \addcontentsline{toc}{subsection}{Uppgift \thetaskcounter{} #1}
  \vspace{1em}~
  \\\normalfont{\large{\bfseries{\hspace{0.5em}Uppgift \thetaskcounter \hspace{1em}#1}}}\\\\
}

% Remove date
\date{}

\hypersetup{
  colorlinks = true,
  linkcolor = blue,
  citecolor = red
}

\lstset{
  language=[Sharp]C,
  basicstyle=\color[rgb]{0.3,0.3,0.3}\ttfamily,
  keywordstyle=\color[rgb]{0,0.5,0.5},
  numberstyle=\color[rgb]{0.7,0.7,0.7},
  commentstyle=\color[rgb]{0.1,0.5,0.1},
  stringstyle=\color[rgb]{0.6,0.1,0.5},
  backgroundcolor=\color[rgb]{0.95,0.95,0.95},
  showstringspaces=false,
  numbers=left,
  breaklines,
  breakatwhitespace,
}

\title{ Labb 2 -- Pixelgrafik }

\author{ Multimedia 7.5 hp VT-14 }
\begin{document}
\maketitle
\vspace{-2em}
%\tableofcontents



\section{Introduktion}
I denna laboration kommer vi att arbeta med pixelgrafik ur två aspekter. Först genom att manuellt skapa pixelgrafik med hjälp av HTML-elementet \texttt{<canvas>} och sedan genom att arbeta med programmet \emph{Adobe Photoshop}.

\section{Inlämning}
Denna laboration består av tre faser där du ska lämna in varje fas i en egen mapp. Din inlämning ska alltså bestå av en .zip-fil eller .rar-fil (inga andra komprimeringsformat är tillåtna!) innehållandes följande (med följande struktur):
  \begin{itemize}
    \item labb2\_fornamn\_efternamn.zip

      \begin{itemize}
        \item uppgift1 (mapp)
          \begin{itemize}
            \item patterns.js (dina färdigskrivna funktioner)
            \item original1.png (som du utgick ifrån)
            \item original2.png (som du utgick ifrån)
            \item original3.png (som du utgick ifrån)
          \end{itemize}

        \item uppgift2 (mapp)
          \begin{itemize}
            \item pattern.psd  (ditt mönster återskapat i Photoshop)
            \item pattern.png  (ovan exporterad till .png)
            \item original.png (som du utgick ifrån)
          \end{itemize}

        \item uppgift3 (mapp)
          \begin{itemize}
	    \item compositeImage.psd (ditt kollage)
            \item compositeImage.jpg (ovan exporterad till .jpg)
            \item originalPicture1.jpg (första orginalbilden)
            \item originalPicture2.jpg (andra orginalbilden)
          \end{itemize}
    \end{itemize}
  \end{itemize}


\pagebreak
\section{Uppgifter}
Nedan följer uppgifterna som resulterar i inlämningarna ovan.




\exercise{Rita med hjälp av javascript och canvas}
  De tre små rutorna ritar sitt innehåll med hjälp av metoderna drawBox1,drawBox2 och drawBox3. Som ni ser ritas för närvarande endast tre rektanglar. 

  \begin{enumerate}
      \item  \href{https://github.com/BinaryHeart/canvas-viewer/archive/master.zip}{ Här}
      \item Öppna index.html i webbläsaren och begrunda att det finns 4 rutor varav om du klickar på en av      de små kopieras innehållet till den stora rutan.
      \item Öppna uppgift.js (i exempelvis notepad++) som finns i uppgiftsmappen och inse att metoderna representerar innehållet till varsin ruta.
      \item Innan laboration bör du läst och förstått hur canvas funkar. Du kan använda detta verktyg för att rita direkt i webbläsaren:   \href{http://www.w3schools.com/html/tryit.asp?filename=tryhtml5_canvas_first}{Canvas på w3schools }
      \item Nu skall du välja tre olika bilder och rita upp dem i de olika rutorna. Ni väljer som tidigare tre bilder här:  \href{http://chrokh.github.io/svg-and-canvas-exercises}{http://chrokh.github.io/svg-and-canvas-exercises}.
    \end{enumerate}

  \exercise{Enkla mönster i Photoshop}
  Denna övning går ut på att göra (nästan) exakt samma sak som i förra övningen, fast nu genom att använda programmet \emph{Adobe Photoshop}.

    \begin{enumerate}
      \item Välj ett nytt mönster.
      \item Skapa ett nytt Photoshop-dokument som är 500x500 pixlar stort.
      \item Återskapa det andra mönstret genom att använda verktygen i Photoshop.
    \end{enumerate}

\exercise{Ett kollage i Photoshop!}
  Du bör nu kunna frilägga ett specifikt objekt från en bild och lägga in den i en
  ny bild dvs göra ett montage. 
  \begin{enumerate}
      \item Ladda hem en bild från internet, det finns arkiv med fria bilder. Alternatitv ta egna foton och använd dem.
      \item Klipp ut minst ett objekt ur ena bilden och lägg in den på den andra bilden. Resultatet skall bli sådant att en ovan användare inte skall kunna se att du manipulerat bilden.
    \end{enumerate}





\end{document}