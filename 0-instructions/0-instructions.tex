\documentclass[12pt]{article}
\usepackage{fullpage}
\usepackage[swedish]{babel}
\usepackage[utf8]{inputenc} % åäö
%\usepackage[T1]{fontenc}
\usepackage{graphicx}
\usepackage{hyperref}
\usepackage{xcolor}
\usepackage{listings}
\usepackage{enumitem}

% No numbering
\setcounter{secnumdepth}{0}

% Counters for tasks & questions
\newcounter{taskcounter}
\setcounter{taskcounter}{0}
\newcounter{stepcounter}
\setcounter{stepcounter}{0}
\newcounter{questioncounter}
\setcounter{questioncounter}{0}

% Tasks
\newcommand{\task}[1]{
  \refstepcounter{taskcounter}
  \addcontentsline{toc}{subsection}{Delmoment \thetaskcounter{} #1}
  \vspace{1em}~
  \\\normalfont{\large{\bfseries{\hspace{0.5em}Delmoment \thetaskcounter \hspace{1em}#1}}}\\\\
}


% Steps
\newenvironment{steps}{ 
  \begin{enumerate}[label=\textbf{Steg \arabic{enumi}:}, leftmargin=5.5em]
    \setcounter{enumi}{\value{stepcounter}}
}{
  \setcounter{stepcounter}{\value{enumi}}
  \end{enumerate}
}

% Questions
\newcommand{\question}[1]{
  \refstepcounter{questioncounter}
  \addcontentsline{toc}{subsection}{Fråga: 4.\thequestioncounter{} #1}
  \vspace{1em}~
  \\\normalfont{\large{\bfseries{\hspace{0.5em}Fråga 4.\thequestioncounter \hspace{1em}#1}}}\\\\
}

% Remove date
\date{}

\hypersetup{
  colorlinks = true,
  linkcolor = blue,
  citecolor = red
}

\lstset{
  language=[Sharp]C,
  basicstyle=\color[rgb]{0.3,0.3,0.3}\ttfamily,
  keywordstyle=\color[rgb]{0,0.5,0.5},
  numberstyle=\color[rgb]{0.7,0.7,0.7},
  commentstyle=\color[rgb]{0.1,0.5,0.1},
  stringstyle=\color[rgb]{0.6,0.1,0.5},
  backgroundcolor=\color[rgb]{0.95,0.95,0.95},
  showstringspaces=false,
  numbers=left,
  breaklines,
  breakatwhitespace,
}

\title{ Information om laborationer }

\author{ Multimedia 7.5 hp VT-14 }
\begin{document}
\maketitle
\vspace{-2em}
%\tableofcontents



\section{Introduktion}
Varje labb i denna kurs håller samma upplägg. Detta dokument berättar om det generella upplägget.

\section{Instruktioner}
Varje labb i denna kurs har ett eget laborationsdokument som du hittar under respektive laborations praktiska inlämningsarea på \emph{Studentportalen}. Varje laboration har dock ytterligare två \textbf{obligatoriska} moment. Det ena vilket behöver utföras \textbf{före} labben (läs mer om detta under rubriken \emph{Förberedelse}) och det andra vilket behöver utföras \textbf{efter}. Dessa sistnämnda moment består av frågeformulär som syftar till att säkerställa inlärning av teori. Dessa frågeformulär besvaras genom \emph{Studentportalen}.

  \paragraph{}
  För att sammanfatta så behöver du alltså avklara \textbf{följande tre moment per labb}.

  \begin{enumerate}
    \item Obligatorisk teoretisk inlämning (före labbtillfället).
    \item Obligatorisk praktisk inlämning (under/efter labbtillfället).
    \item Obligatorisk teoretisk inlämning (efter labbtillfället).
  \end{enumerate}

Laborationerna ska utföras och lämnas in \textbf{individuellt} om inget annat anges i specifika laborationsdokumenten.

\section{Inlämning}
Det är viktigt att lämna in i tid, och att lämna in enligt instruktionerna. De teoretiska inlämningarna utförs alla på samma sätt (genom frågeformulär på Studentportalen), oavsett labb. De praktiska kan dock komma att variera. Det är då viktigt att du lämnar in i enlighet med det sätt som beskrivs i respektive labbinstruktionsdokument.

\section{Förberedelse}
Det förväntas att ni tillgodosett er förberedelsematerialet (som ni hittar under respektive teoretiska pre-labb-inlämning) innan varje laborationstillfälle. Anledningen till att de teoretiska inlämningarna innan labbtillfällena finns är alltså att vi inte vet huruvida ni tar del av förberedelsematerialet eller ej.

Men, \textbf{vi förväntar oss att ni tar del av förberedelsematerialet innan varje laboration}, även om ni lyckas svara på frågorna utan att göra det. För att er lärandeprocess ska fortskrida så smidigt som möjligt, och för att labbhandledarna ska kunna hjälpa er under laborationerna är det viktigt att ni faktiskt tar del av materialet innan respektive labb.

\section{Lycka till!}


\end{document}