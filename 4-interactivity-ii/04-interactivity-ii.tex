\documentclass[12pt]{article}
\usepackage{fullpage}
\usepackage[swedish]{babel}
\usepackage[utf8]{inputenc} % åäö
%\usepackage[T1]{fontenc}
\usepackage{graphicx}
\usepackage{hyperref}
\usepackage{xcolor}
\usepackage{listings}
\usepackage{enumitem}



% No numbering
\setcounter{secnumdepth}{0}

% Counters for tasks & questions
\newcounter{taskcounter}
\setcounter{taskcounter}{0}
\newcounter{stepcounter}
\setcounter{stepcounter}{0}
\newcounter{questioncounter}
\setcounter{questioncounter}{0}

% Exercises
\newcommand{\exercise}[1]{
  \refstepcounter{taskcounter}
  \addcontentsline{toc}{subsection}{Uppgift \thetaskcounter{} #1}
  \vspace{1em}~
  \\\normalfont{\large{\bfseries{\hspace{0.5em}Uppgift \thetaskcounter \hspace{1em}#1}}}\\\\
}

% Remove date
\date{}

\hypersetup{
  colorlinks = true,
  linkcolor = blue,
  citecolor = red
}

\lstset{
  language=[Sharp]C,
  basicstyle=\color[rgb]{0.3,0.3,0.3}\ttfamily,
  keywordstyle=\color[rgb]{0,0.5,0.5},
  numberstyle=\color[rgb]{0.7,0.7,0.7},
  commentstyle=\color[rgb]{0.1,0.5,0.1},
  stringstyle=\color[rgb]{0.6,0.1,0.5},
  backgroundcolor=\color[rgb]{0.95,0.95,0.95},
  showstringspaces=false,
  numbers=left,
  breaklines,
  breakatwhitespace,
}


\title{ Labb 4 -- Interaktivitet II }

\author{ Multimedia 7.5 hp VT-14 }
\begin{document}
\maketitle
\vspace{-2em}
%\tableofcontents



\section{Introduktion}
I denna laboration kommer vi att arbeta mer med javascript för att skapa interaktivitet. Tanken är nu att vi fördjupar oss i alla delarna -- javascript, html såväl som css. Men under denna labb arbetar du med fördel med \href{http://jquery.com/}{jQuery}. När man jobbar med nya tekniker är det ofta troligt att man behöver stanna upp och söka information, så var inte rädd för att aktivt använda internet till att leta information.

\section{Inlämning}
Din inlämning ska bestå av en .zip-fil eller .rar-fil (inga andra komprimeringsformat är tillåtna!) innehållandes följande (med följande struktur):
  \begin{itemize}
    \item labb2\_fornamn\_efternamn.zip

      \begin{itemize}
        \item uppgift1 (mapp)
          \begin{itemize}
            \item buttons.js
            \item buttons.css 
            \item index.html
          \end{itemize}
	      \item uppgift2 (mapp)
          \begin{itemize}
            \item memory.js
            \item memory.css
            \item index.html
          \end{itemize}
    \end{itemize}
  \end{itemize}

  
  \subsection{Kodstandard}
    \begin{itemize}
      \item Ingen Javascript eller CSS ska placeras ``inline'' i något htmldokument. All javascriptkod skall alltså skrivas i \texttt{.js}-filer och css i \texttt{.css}-filer.
      \item All kod ska vara korrekt indenterad! (Läs mer om indentering på \href{http://htmlhunden.se}{htmlhunden} om du är osäker)
      \item Ladda in kända externa bibliotek (i detta fall jQuery) ifrån en CDN (ex: \href{http://code.jquery.com/}{jQuery CDN}) istället för att spara biblioteket i ditt projekt.
    \end{itemize}


\pagebreak
\section{Uppgifter}
Nedan följer uppgifterna som resulterar i inlämningarna ovan.



  \exercise{}



  \exercise{Rutnät av bilder}
 I denna uppgift skall ni skapa en sida som har ett rutnät/schackbräde av bilder. Bilder finns att hämta \href{http://http://placekitten.com/}{här} och HÄR OSV. 


  \begin{enumerate}
    \item Börja med att skapa en .html-fil 
    \item Skapa sedan en \texttt{.js}-fil där du kommer att skapa metoder som du behöver. 
    \item Skapa sedan en \texttt{.css}-fil där all formatering skall finnas! 
 \end{enumerate}


  \paragraph{Krav}
    \begin{itemize}
      \item Rutnätet skall göras dynamiskt, det är alltså inte ok att skapa ett rutnät som är av en hårdkodad storlek. Således kan det t.ex. vara bra att abstrahera ditt program till en metod som du förslagsvis kallar \texttt{drawGrid} och som tar två siffror: \texttt{rows} och \texttt{cols}. Ditt rutnät ska alltså kunna skapas som 2x2 såväl som 20x30.
     \item Rutnätet ska alltid täcka 100\% av fönstrets bredd och 100\% av fönstrets höjd.
    \end{itemize}




  \exercise{Spelet Memory}
  Denna uppgift går ut på att skapa en enklare variant av spelet memory. Spelplanen består av ett antal bilder, i par. Spelet går ut på att man ska välja två bilder, om de två bilderna är likadana så får man ta bort dem ifrån spelplanen. När alla bilder är borta ifrån spelplanen är spelet över och man har vunnit!

  \paragraph{}
  I vanliga fall spelar man med uppochnedåtvända bilder, för att spelet förstås inte ska vara busenkelt. Detta kommer vi att strunta i när vi bygger vårt eget Memory. Alla brickor i ditt memory-spel ska alltså hela tiden vara ``uppåtvända''. Man ser alltså alltid alla bilder, och spelet går helt enkelt ut på att klicka på par.

  \paragraph{Krav}
  \begin{itemize}
    \item Din spelplan måste vara minst 3x2
    \item När en korrekt matchning är gjord ska båda brickorna tas bort
    \item När en inkorrekt matching görs ska ett meddelande visas för användaren
    \item När spelet är slut ska ett meddelande visas för användaren
    \item De bilder du använder måste vara av rimlig storlek, både i dimensioner och i ``tyngd''.
  \end{itemize}

  \paragraph{Förslag på tillvägagångssätt}
  \begin{enumerate}
    \item Skapa en HTML, en CSS, en JS-fil, och ``koppla ihop'' dem.
    \item Skriv ut dina bilder i din HTML.
    \item Hitta ett sätt att identifiera par, t.ex. genom \texttt{ID}:n och/eller \texttt{class}:er.
    \item Fundera över vidare steg som behöver tas och bryt ned i delproblem.
  \end{enumerate}








\end{document}