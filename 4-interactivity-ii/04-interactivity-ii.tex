\documentclass[12pt]{article}
\usepackage{fullpage}
\usepackage[swedish]{babel}
\usepackage[utf8]{inputenc} % åäö
%\usepackage[T1]{fontenc}
\usepackage{graphicx}
\usepackage{hyperref}
\usepackage{xcolor}
\usepackage{listings}
\usepackage{enumitem}



% No numbering
\setcounter{secnumdepth}{0}

% Counters for tasks & questions
\newcounter{taskcounter}
\setcounter{taskcounter}{0}
\newcounter{stepcounter}
\setcounter{stepcounter}{0}
\newcounter{questioncounter}
\setcounter{questioncounter}{0}

% Exercises
\newcommand{\exercise}[1]{
  \refstepcounter{taskcounter}
  \addcontentsline{toc}{subsection}{Uppgift \thetaskcounter{} #1}
  \vspace{1em}~
  \\\normalfont{\large{\bfseries{\hspace{0.5em}Uppgift \thetaskcounter \hspace{1em}#1}}}\\\\
}

% Remove date
\date{}

\hypersetup{
  colorlinks = true,
  linkcolor = blue,
  citecolor = red
}

\lstset{
  language=[Sharp]C,
  basicstyle=\color[rgb]{0.3,0.3,0.3}\ttfamily,
  keywordstyle=\color[rgb]{0,0.5,0.5},
  numberstyle=\color[rgb]{0.7,0.7,0.7},
  commentstyle=\color[rgb]{0.1,0.5,0.1},
  stringstyle=\color[rgb]{0.6,0.1,0.5},
  backgroundcolor=\color[rgb]{0.95,0.95,0.95},
  showstringspaces=false,
  numbers=left,
  breaklines,
  breakatwhitespace,
}


\title{ Labb 4 -- Interaktivitet II }

\author{ Multimedia 7.5 hp VT-14 }
\begin{document}
\maketitle
\vspace{-2em}
%\tableofcontents



\section{Introduktion}
I denna laboration kommer vi att arbeta mer med javascript för att skapa interaktivitet. Tanken är nu att vi fördjupar oss i alla delarna javascript, html och css. När man jobbar med nya tekniker gäller alltid att man behöver ibland gå tillbaka och aktivt söka information. 

\section{Inlämning}
Din inlämning ska bestå av en .zip-fil eller .rar-fil (inga andra komprimeringsformat är tillåtna!) innehållandes följande (med följande struktur):
  \begin{itemize}
    \item labb2\_fornamn\_efternamn.zip

      \begin{itemize}
        \item uppgift1 (mapp)
          \begin{itemize}
            \item grid.js 
            \item index.html
	    \item grid.css
          \end{itemize}
        \item uppgift2 (mapp)
          \begin{itemize}
            \item randomizer.js
            \item index.html
          \end{itemize}
	      \item uppgift3 (mapp)
          \begin{itemize}
            \item menu.js
            \item menu.css
            \item index.html
          \end{itemize}
    \end{itemize}
  \end{itemize}

  
  \subsection{Kodstandard}
    \begin{itemize}
      \item Ingen Javascript eller CSS ska placeras ``inline'' i något htmldokument. All javascriptkod skall alltså skrivas i \texttt{.js}-filer och css i \texttt{.css}-filer.
      \item All kod ska vara korrekt indenterad! (Läs mer om indentering på \href{http://htmlhunden.se}{htmlhunden} om du är osäker)
    \end{itemize}


\pagebreak
\section{Uppgifter}
Nedan följer uppgifterna som resulterar i inlämningarna ovan.



  \exercise{Rutnät av bilder}
 I denna uppgift skall ni skapa en sida som har ett rutnät/schackbräde av bilder. Bilder finns att hämta \href{http://http://placekitten.com/}{här} och HÄR OSV. 


  \begin{enumerate}
    \item Börja med att skapa en .html-fil 
    \item Skapa sedan en \texttt{.js}-fil där du kommer att skapa metoder som du behöver. 
    \item Skapa sedan en \texttt{.css}-fil där all formatering skall finnas! 
 \end{enumerate}


  \paragraph{Krav}
    \begin{itemize}
      \item Rutnätet skall göras dynamiskt, det är alltså inte ok att skapa ett rutnät som är av en fix storlek utan rutnätet skall skapas med javaskript.
     \item Sidan skall vara skalbar och med det menas att om fönstret blir mindre så skall bilderna anpassas efter det. 
    \end{itemize}






  \exercise{Slumpmässig färg}
  I denna övning ska vi träna på att förändra CSS-egenskaper genom JavaScript. Målet är en helt tom sida med en knapp/länk. När man trycker på länken så ska hela sidans bakgrundsfärg ändras. En ny bakgrundsfärg ska slumpas fram vid varje knapptryckning.

  Tänk på att den här uppgiften kan delas upp i (t.ex.) följande delprpblem:

  \begin{itemize}
    \item ``Lyssna'' på knapptryckningen
    \item Slumpa fram tal
    \item Slumpa fram en färg
    \item Ändra bakgrundsfärgen
  \end{itemize}

  Tips: Det kan underlätta att sätta färgerna genom \texttt{rgb(x,y,z)} istället för att använda hexadecimal notation (e.g. \#333333).




\exercise{Interaktiv meny}
  Denna övning ger dig lite större frihet än de tidigare. Målet med denna övning är att producera en \texttt{.html}-sida innehållandes tre menyknappar. Sidan ska även innehålla tre informationsrutor (ex. \texttt{<div>:ar}). När sidan laddas ska alla informationsrutorna utom den första vara dold. När användaren sedan klickar på respektive meny-knapp så ska respektive informationsruta visas, och övriga döljas.

  \paragraph{}
  Endast seriösa försök beaktas som inlämningar. Skapa en vision i ditt huvud kring hur sidan borde se ut för att på lättast möjliga sätt kommunicera sitt syfte (navigationen) till användaren och exekvera din vision med hjälp av \texttt{css}.

  \paragraph{}
  Tips: Försök att, precis som i förra uppgiften, bryta upp problemet i delproblem, och använd dig av funktioner för att strukturera din Javascript-kod!






\end{document}