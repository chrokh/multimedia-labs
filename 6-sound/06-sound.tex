\documentclass[12pt]{article}
\usepackage{fullpage}
\usepackage[swedish]{babel}
\usepackage[utf8]{inputenc} % åäö
%\usepackage[T1]{fontenc}
\usepackage{graphicx}
\usepackage{hyperref}
\usepackage{xcolor}
\usepackage{listings}
\usepackage{enumitem}



% No numbering
\setcounter{secnumdepth}{0}

% Counters for tasks & questions
\newcounter{taskcounter}
\setcounter{taskcounter}{0}
\newcounter{stepcounter}
\setcounter{stepcounter}{0}
\newcounter{questioncounter}
\setcounter{questioncounter}{0}

% Exercises
\newcommand{\exercise}[1]{
  \refstepcounter{taskcounter}
  \addcontentsline{toc}{subsection}{Uppgift \thetaskcounter{} #1}
  \vspace{1em}~
  \\\normalfont{\large{\bfseries{\hspace{0.5em}Uppgift \thetaskcounter \hspace{1em}#1}}}\\\\
}

% Remove date
\date{}

\hypersetup{
  colorlinks = true,
  linkcolor = blue,
  citecolor = red
}

\lstset{
  language=[Sharp]C,
  basicstyle=\color[rgb]{0.3,0.3,0.3}\ttfamily,
  keywordstyle=\color[rgb]{0,0.5,0.5},
  numberstyle=\color[rgb]{0.7,0.7,0.7},
  commentstyle=\color[rgb]{0.1,0.5,0.1},
  stringstyle=\color[rgb]{0.6,0.1,0.5},
  backgroundcolor=\color[rgb]{0.95,0.95,0.95},
  showstringspaces=false,
  numbers=left,
  breaklines,
  breakatwhitespace,
}


\title{ Labb 6 -- Animation }

\author{ Multimedia 7.5 hp VT-14 }
\begin{document}
\maketitle
\vspace{-3.5em}
%\tableofcontents


% FÖRBEREDELSE-MATERIAL
% http://www.youtube.com/watch?v=VJufqV6P-xY (Scrolling background header)


\section{Introduktion}
I denna laboration kommer vi att arbeta med författande och redigering av ljud, såväl som inbäddning av ljud i webbsidor.

\section{Inlämning}
Din inlämning ska bestå av en .zip-fil eller .rar-fil (inga andra komprimeringsformat är tillåtna!) innehållandes följande (med följande struktur):
  \begin{itemize}
    \item \texttt{labb6\_fornamn\_efternamn.zip}
      \vspace{-0.5em}
      \begin{itemize}
        \item \texttt{uppgift1} (mapp)
          \begin{itemize}
            \item \texttt{original.mp3}
            \item \texttt{effect1-<name-of-effects>.mp3} (utan \texttt{<>}-tecknen)
            \item \texttt{effect2-<name-of-effects>.mp3} (utan \texttt{<>}-tecknen)
            \item \texttt{effect3-<name-of-effects>.mp3} (utan \texttt{<>}-tecknen)
          \end{itemize}
        \item \texttt{uppgift2} (mapp)
          \begin{itemize}
            \item \texttt{story.mp3}
            \item \texttt{story.ogg}
            \item \texttt{story.txt}
          \end{itemize}
        \item \texttt{uppgift3} (mapp)
          \begin{itemize}
            \item \texttt{index.html}
            \item \texttt{audio.css}
          \end{itemize}
    \end{itemize}
  \end{itemize}

  \subsection{Krav}
  \begin{itemize}
    \item Ljudfilen \texttt{story.[mp3/ogg]} skall alltså \textbf{endast} lämnas in i \texttt{uppgift 2}. I \texttt{uppgift 3} ska du referera till dina ljudfiler som ligger i mappen \texttt{uppgift2} genom \texttt{relativ länkning}.
    \item All kod ska vara korrekt indenterad. Läs mer på \href{http://htmlhunden.se}{HTMLHunden} om du är osäker på hur man indenterar.
    \item Alla ljudfiler måste hålla \textbf{rimliga} filstorlekar (\texttt{< 3 MB}).
  \end{itemize}



\pagebreak
\section{Uppgifter}
Nedan följer uppgifterna som resulterar i inlämningarna ovan.




  \exercise{Ljudredigering -- Audacity}
  I denna uppgift ska vi arbeta med ljudredigering. Poängen är helt enkelt att experimentera med ett antal effekter. Uppgiften går ut på att klippa ned en ljudfil till en 10-sekunders-slinga och sedan applicera tre olika effekter.

  \begin{enumerate}
    \item Leta rätt på en ljudfil på nätet (exempelvis ifrån \href{http://www.freesound.org/}{freesound.org}) och ladda hem den. Ljudfilen måste vara längre än 10 sek.
    \item Öppna ljudfilen i \texttt{Audacity} och korta ner den till 10 sek genom att klippa bort de delar du inte vill ha.
    \item Spara ditt \texttt{Audacity}-projekt.
    \item Exportera den nedkortade versionen som \texttt{.mp3}. Detta är den fil som du ska lämna in som \texttt{original.mp3}.
    \item Nästa uppdrag är nu modifiera denna 10 sekunders ljudfil, tre gånger. Börja alltid med att göra en kopia av ditt \texttt{Audacity}-projekt så att du inte arbetar med redan komprimerade filer. **
    \item För varje version ska du alltså applicera en eller flera effekter såsom t.ex. \texttt{echo}, \texttt{reverb}, \texttt{fade in/out}, \texttt{reverse} etc.
    \item Exportera varje modifierad version av din ljudfil som \texttt{.mp3}.
  \end{enumerate}

  \paragraph{**}
  Varför inte? Det här handlar om samma problem som med \texttt{.jpeg}-komprimering. Om du sparar en fil i ett format med förstörande komprimering, öppnar filen igen, gör ändringar, och sedan sparar den i ett format med förstörande komprimering igen, så kommer vi successivt att försämra filen.


  \pagebreak
  \exercise{Ljudförfattning -- Garage Band}
  I denna uppgift ska vi arbeta med ljudförfattning. Mer specifikt så ska vi skapa en ljudhistoria. Tänk: radioteater eller podcasts med berättelser. Ditt uppdrag är alltså att först skriva en kort historia om någonting, precis vad som helst. Sedan skall du ljudsätta denna historia med bakgrundsljud och effekter. Om historian handlar om en dag i parken så kanske man hör grodor som kvackar, vatten som porlar, och vinden som blåser i träden. När någon går kanske vi hör fotsteg, och när en bil åker förbi motorn som rumblar o.s.v.
  \paragraph{}
  I den här uppgiften arbetar du med fördel i \texttt{Garage Band}, som redan innehåller en mängd ljud som är passande för t.ex. podcasts.

  \begin{enumerate}
    \item Skriv ihop din historia.
    \item Skapa ett nytt \texttt{Garage Band}-projekt.
    \item Läs in din historia (alltså spela in någon som läser upp den) och lägg in i ditt projekt. (Frivilligt steg).
    \item Lägg sedan till passande ljud för din historia. Använd antingen de redan existerande ljuden i \texttt{Garage Band} eller hitta nya genom internet.
    \item När du är klar: exportera din ljudberättelse som \texttt{.mp3} och som \texttt{.ogg}.
  \end{enumerate}



  \exercise{Ljuddistribution -- HTML5 Audio}
  Vi ska nu prova på ett sätt att bädda in ljud i webbsidor. Detta genom att använda oss avb \texttt{HTML5}-elementet \texttt{<audio>}.

  \begin{enumerate}
    \item Skapa en \texttt{.html}-, en \texttt{.css}-fil och ``koppla ihop dem''.
    \item Använd \texttt{<audio>}-elementet för att bädda in dina ljudfiler.
  \end{enumerate}

  \paragraph{Krav}
  \begin{itemize}
    \item Kom ihåg att ge båda versionerna (formaten) av din ljudfil som alternativ till \texttt{<audio>}-elementet. Alltså både \texttt{.mp3} och \texttt{.ogg}.
    \item Kom ihåg att du inte ska kopiera dina ljudfiler ifrån \texttt{uppgift 2} till \texttt{uppgift 3}. Istället ska du använda \texttt{relativa url}:er för att länka till filerna i mappen för \texttt{uppgift 2}.
  \end{itemize}


\end{document}